\section{Increment / Decrement Operator}

I have always been confused by the way increment/decrement operators work. I am familiar with what they do: to increment or decrement values. What I never quite understood was the difference between post and pre-increment or decrement.
To illustrate this important, somehow hidden point, lets review an example:

\begin{lstlisting}[language=C++]
int i = 0;
if(i++ == 1) {
	std::cout << "i++ is equal to 1" << std::endl;
}
//here, the value of i is 1 for sure
if(++i == 2) {
	std::cout << "++i is equal to 2" << std::endl;
}
\end{lstlisting}

The output of the above example is:
\begin{quote}
++i is equal to 2
\end{quote}

To understand what is going on here, lets see what these operators actually do:
\begin{itemize}
\item \textbf{Pre-increment and pre-decrement} - Increments or decrements the value of the object and returns a reference to the result.
\item \textbf{Post-increment and post-decrement} - Creates a copy of the object, increments or decrements the value of the object and returns the copy from before the increment-decrement operation.
\end{itemize}

\subsection{Special Notes}
\begin{itemize}
\item \textbf{Performance} - Due to the fact that post-increment and post-decrement operations involve creating a temporary copy object, they are less efficient than pre-increment and pre-decrement operators in contexts where the returned value is not used.
\item \textbf{Undefined Behavior} - Due to sequencing order violations. \emph{To-be-explained}
\end{itemize}
